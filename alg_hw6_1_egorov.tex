\documentclass[a4paper]{article}
\usepackage[utf8]{inputenc}
\usepackage[russian]{babel}
\usepackage[T1]{fontenc}
\usepackage{amsmath}
\usepackage{delarray}
\usepackage{amsfonts}
\usepackage{amssymb}
\usepackage{graphicx}
\author{Борис Егоров}
\title{ДЗ6 задача 1 алгоритмы}
\begin{document}
	\maketitle
	\begin{large}
	обозначим события: $A$ - выпадение $n$ орлов, $B_0$ - выбрана неправильная монета, $B_1$ - выбрана правильная монета;\\
	нужно найти $n$, начиная с которого $P(B_0|A)\ge0.95$;\\
	очевидно, что $B_0$ и $B_1$ несовместны, поэтому можно применить формулу полной вероятности:\\\\
	$
	P(B_0|A)=\frac{P(B_0A)}{P(A)}=\frac{P(A|B_0)P(B_0)}{P(A|B_0)P(B_0)+P(A|B_1)P(B_1)}=\\\\
	\frac{0.9^n\cdot0.5}{0.9^n\cdot0.5+0.5^n\cdot0.5}=
	\frac{0.9^n}{0.9^n+0.5^n};\\\\
	\frac{0.9^n}{0.9^n+0.5^n}\ge0.95\\\\
	0.9^n\ge0.95(0.9^n+0.5^n)\\\\
	0.9^n\cdot0.05\ge0.5^n\cdot0.95\\\\
	0.9^n\ge0.5^n\cdot19\\\\
	1.8^n\ge19
	$\\\\
	найдем $n$ подбором: $1.8^6>19$ и $1.8^5<19$;\\
	ответ: начиная с 6.
	\end{large}
\end{document}