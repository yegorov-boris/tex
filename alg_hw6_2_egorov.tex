\documentclass[a4paper]{article}
\usepackage[utf8]{inputenc}
\usepackage[russian]{babel}
\usepackage[T1]{fontenc}
\usepackage{amsmath}
\usepackage{delarray}
\usepackage{amsfonts}
\usepackage{amssymb}
\usepackage{graphicx}
\author{Борис Егоров}
\title{ДЗ6 задача 2 алгоритмы}
\begin{document}
	\maketitle
	\begin{large}
	$PAM=A+(1-4a)E$;\\
	$E$ - единичная матрица размера 4 на 4;\\
	$A$ - матрица размера 4 на 4, каждый элемент которой равен $a$;\\
	так как $E^n=E$, то $((1-4a)E)^n=(1-4a)^nE$;\\\\
	обозначим $p_{ij}^{(k)}$ элемент в i-й строке и j-м столбце матрицы\\
	$A^k,k\in\mathbb{N}$;\\
	так как все элементы матрицы $A$ равны, то и все элементы матрицы $A^n$ будут равны между собой: $p_{ij}^{(k)}=p^{(k)}$;\\
	так как $A^{k+1}=A^kA$ и $A$ размера 4 на 4, то $p^{(k+1)}=4ap^{(k)}$;\\
	возьмем матрицу $A$ и умножим $n-1$ раз на $A$:\\
	при 1-м умножении $p^{(1)}=a$, поэтому после n-1-го умножения получим $p^{(n)}=4^{n-1}a^n$;\\\\
	$
	PAM^n=(A+(1-4a)E)^n=\sum_{k=0}^{n}\binom{n}{k}A^{n-k}(1-4a)^kE=\\\\
	A^n+(1-4a)^nE+\sum_{k=1}^{n-1}\binom{n}{k}A^{n-k}(1-4a)^kE=\\\\
	A^n+(1-4a)^nE+\sum_{k=1}^{n-1}\binom{n}{k}(1-4a)^kA^{n-k}
	$;\\\\
	рассмотрим элемент на главной диагонали матрицы $PAM^n$:\\\\
	$
	PAM_{ii}^{(n)}=4^{n-1}a^n+(1-4a)^n+\sum_{k=1}^{n-1}\binom{n}{k}4^{n-k-1}a^{n-k}(1-4a)^k=\\\\
	\frac{1}{4}(4^na^n+\sum_{k=1}^{n-1}\binom{n}{k}4^{n-k}a^{n-k}(1-4a)^k)+(1-4a)^n=\\\\
	\frac{1}{4}(4^na^n+(1-4a)^n+\sum_{k=1}^{n-1}\binom{n}{k}4^{n-k}a^{n-k}(1-4a)^k)+\frac{3}{4}(1-4a)^n=\\\\
	\frac{1}{4}\sum_{k=0}^{n}\binom{n}{k}(4a)^{n-k}(1-4a)^k)+\frac{3}{4}(1-4a)^n=\\\\
	\frac{1}{4}(4a+1-4a)^n+\frac{3}{4}(1-4a)^n=\\\\
	\frac{1}{4}+\frac{3}{4}(1-4a)^n
	$;\\\\
	рассмотрим элемент не на главной диагонали матрицы $PAM^n$:\\
	каждое слагаемое вида $A^{n-k}(1-4a)^kE,k\ne n$ - это матрица с одинаковыми элементами;\\
	при $k=n$ получим $A^0(1-4a)^nE=E(1-4a)^nE=(1-4a)^nE$ - диагональная матрица;\\
	то есть у каждого слагаемого суммы $\sum_{k=0}^{n}\binom{n}{k}A^{n-k}(1-4a)^kE$ элементы не на главной диагонали равны между собой;\\
	обозначим $t=PAM_{ij}^{(n)},j\ne i$;\\
	так как $PAM^n$ - матрица перехода, то $\sum_{j=1}^{4}PAM_{ij}^{(n)}=1$, тогда\\\\
	$
	3t+\frac{1}{4}+\frac{3}{4}(1-4a)^n=1;\\\\
	t=\frac{1}{3}(1-\frac{1}{4}-\frac{3}{4}(1-4a)^n)=\frac{1}{4}-\frac{1}{4}(1-4a)^n
	$\\\\
	ответ:\\\\
	$
	PAM_{ij}^{(n)}=\left.
	\begin{array}{l}
		\frac{1}{4}+\frac{3}{4}(1-4a)^n,i-j\\\\
		\frac{1}{4}-\frac{1}{4}(1-4a)^n,i\ne j
	\end{array}
	\right\}
	$
	\end{large}
\end{document}