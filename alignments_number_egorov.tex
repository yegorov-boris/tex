\documentclass[a4paper]{article}
\usepackage[utf8]{inputenc}
\usepackage[russian]{babel}
\usepackage[T1]{fontenc}
\usepackage{amsmath}
\usepackage{amsfonts}
\usepackage{amssymb}
\usepackage{graphicx}
\author{Борис Егоров}
\title{число выравниваний}
\begin{document}
	\maketitle
	\begin{large}
	количество всех возможных выравниваний для двух строк длины $n$ и $m$:\\
		\\
		$g(n, m)={n+m \choose n}={n+m \choose m}$\\
		\\
		доказательство:\\
		\\
		рассмотрим точечную матрицу сходства из $m$ строк и $n$ столбцов\\
		найдем число всех возможных переходов "вправо вниз" (соответствует букве над буквой, без гэпа)\\
		\\
		таких переходов может быть от $0$ $m$\\
		\\
		$0 \le k \le m$ переходов расположены в $k$ из $m$ строках (в одной строке не может быть больше одного перехода, так как одну позицию последовательности нельзя прочитать больше 1 раза)\\
		\\
		внутри $k$ строк, выбранных одним из ${m \choose k}$ способов, каждый из $k$ переходов находится в одном из $n$ столбцов, но каждый следующий переход должен быть правее предыдущего\\
		\\
		такие $k$ переходов из $n$ различных столбцов можно выбрать ${n \choose k}$ способами\\
		\\
		тогда $g(n,m)=\sum_{k=0}^{m} {m \choose k}{n \choose k}$\\
		\\
		с учетом тождества Вандермонда ${m+n \choose r}=\sum_{k=0}^{r} {m \choose k}{n \choose r-k}$\\
		\\
		перепишем $g(n,m)=\sum_{k=0}^{m} {n \choose k}{m \choose m-k}={m+n \choose m}$
	\end{large}
\end{document}