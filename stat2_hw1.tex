\documentclass{article}
\usepackage[utf8]{inputenc}
\usepackage{amsmath,amssymb}
\usepackage[russian]{babel}
\author{Борис Егоров}
\title{ДЗ1 статистика}
\begin{document}
	\maketitle
	\begin{large}
	\section*{1}
	для непрерывного равномерного распределения на $[-a,a]$ матожидание $EX=0$, дисперсия $DX=\frac{a^2}{3}$, тогда коэффициент эксцесса\\
	$
	\gamma_2=\frac{\mu_4}{\sigma^4}-3=\\
	\frac{E(X-EX)^4}{D^2}-3=\\
	\frac{EX^4}{a^4/9}-3=\\
	\frac{9EX^4}{a^4}-3;\\
	EX^4=\frac{1}{2a}\int_{-a}^{a}x^4dx=\\
	\frac{1}{2a}(\frac{a^5}{5}-\frac{(-a)^5}{5})=\\
	\frac{a^4}{5};\\
	\gamma_2=\frac{9a^4}{5a^4}-3=\frac{9}{5}-3=-\frac{6}{5}
	$\\
	коэффициент эксцесса в данном случае не зависит от формы плотности;\\\\
	для распределения Лапласа коэффициент эксцесса равен 3 независимо от параметров $\alpha$ и $\beta$.
	\section*{2}
	утверждение, что эмпирическая функция распределения $F_n(y)$ - состоятельная оценка генеральной функции распределения $F(y)$ равносильно тому, что $F_n$ сходится по вероятности к $F(y)$;\\
	$F_n(y)=\frac{1}{n}\sum_{i=1}^{n}I(X_i<y)$;\\
	случайные величины $I(X_1<y),...,I(X_n<y)$ независимы и одинаково распределены;\\
	$EI(X_i<y)=F(y)$ (найдено в задаче 4a.),\\
	поэтому $EI(X_i<y)\le1$ - матожидания случайных величин $X_1,...,X_n$ существуют и конечны,\\
	тогда выполняется ЗБЧ Хинчина\\
	$\frac{1}{n}\sum_{i=1}^{n}I(X_i<y) \overset{p}{\to} EI(X_i<y)$ при $n\rightarrow\infty$, тогда\\
	$F_n(y)$ сходится по вероятности к $F(y)$
	\section*{3}
	выборка - это набор случайных величин $X_1,...,X_n$, до эксперимента их значения неизвестны;\\
	какого-то из элементарных исходов может не быть в конкретной реализации выборки.
	\section*{4}
	\subsection*{a}
	$
	EF_n(y)=E\frac{1}{n}\sum_{i=1}^{n}I(X_i<y)=\\
	\frac{1}{n}E\sum_{i=1}^{n}I(X_i<y)=\\
	\frac{1}{n}\sum_{i=1}^{n}EI(X_i<y);
	$\\
	так как $X_1,...,X_n$ одинаково распределены,\\
	то $\sum_{i=1}^{n}EI(X_i<y)=nEI(X_i<y)$, тогда\\
	$
	EF_n(y)=\frac{1}{n}\cdot nEI(X_i<y)=EI(X_i<y)=\\
	0P(X_i\ge y)+1P(X_i<y)=P(X_i<y)=F(y)
	$
	\subsection*{b}
	$
	DF_n(y)=D\frac{1}{n}\sum_{i=1}^{n}I(X_i<y)=\\
	\frac{1}{n^2}D\sum_{i=1}^{n}I(X_i<y);
	$\\
	так как $X_1,...,X_n$ независимы и одинаково распределены, то\\
	$
	\frac{1}{n^2}D\sum_{i=1}^{n}I(X_i<y)=\frac{1}{n^2}\cdot nDI(X_i<y)=\frac{1}{n}DI(X_i<y);
	$\\
	$DI(X_i<y)=E(I(X_i<y))^2-(EI(X_i<y))^2$;\\
	подставим матожидание $I(X_i<y)$, найденное в пункте a.:\\
	$
	DI(X_i<y)=E(I(X_i<y))^2-(F(y))^2;\\
	E(I(X_i<y))^2=0^2P(X_i\ge y)+1^2P(X_i<y)=P(X_i<y)=F(y);\\
	DI(X_i<y)=F(y)-(F(y))^2=F(y)(1-F(y));\\
	DF_n(y)=\frac{F(y)(1-F(y))}{n}
	$
	\section*{6}
	коэффициент эксцесса $\gamma_2=\frac{\mu_4}{\sigma^4}-3$\\
	$\sigma^4=D^2$, где $DX=E(X-EX)^2$ - дисперсия;\\
	так как дисперсия неотрицательна, то $\sigma^4\ge0$;\\
	но так как $\sigma^4$ в знаменателе, то $\sigma^4>0$;\\
	$\mu_4=E(X-EX)^4$\\
	так как вероятность неотрицательна и $E(X-EX)^4\ge0$, то $\mu_4\ge0$,\\
	поэтому $\frac{\mu_4}{\sigma^4}\ge0$ и $\gamma_2\ge-3$;\\
	рассмотрим распределение Бернулли с параметром $p$:\\
	$q=1-p$\\
	$\sigma^4=D^2=(pq)^2=p^2q^2$\\
	$EX=p$\\
	$
	\mu_4=E(X-EX)^4=\\
	E(X-p)^4=\\
	P(X=0)(0-p)^4+P(X=1)(1-p)^4=\\
	qp^4+pq^4=\\
	pq(p^3+q^3);\\
	\gamma_2=\frac{pq(p^3+q^3)}{p^2q^2}-3=\\
	\frac{p^3+q^3}{pq}-3;
	$\\
	пусть $p\rightarrow0$, тогда $q\rightarrow1$ и $\gamma_2\rightarrow+\infty$;\\
	$\gamma_2\in[-3,+\infty)$
	\end{large}
\end{document}