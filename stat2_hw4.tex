\documentclass{article}
\usepackage[utf8]{inputenc}
\usepackage{amsmath,amssymb}
\usepackage{amsfonts}
\usepackage{newpxmath,newpxtext}
\usepackage[russian]{babel}
\usepackage{bm}
\usepackage[T1]{fontenc}
\author{Борис Егоров}
\title{ДЗ4 статистика}

\makeatletter
\newcommand{\oset}[3][0ex]{%
	\mathrel{\mathop{#3}\limits^{
			\vbox to#1{\kern-2\ex@
				\hbox{$\scriptstyle#2$}\vss}}}}
\makeatother

\begin{document}
	\maketitle
	\begin{large}
	\section*{1}
	\subsection*{пункт a}
	выборка $\bm{X}=(X_1,...,X_n)$;\\\\
	критерий
	$
	\delta_n=\delta_n(\bm{X})=\left\{ {\begin{array}{l}
		H_0, \oset[-.3ex]{-}{X}>3.5+1/\sqrt{n} \\
		H_1, \oset[-.3ex]{-}{X}\le 3.5+1/\sqrt{n} \\
		\end{array} } \right.
	$\\\\\\
	ошибка первого рода $\alpha_1(\delta_n)$\\
	ошибка второго рода $\alpha_2(\delta_n)$\\\\
	пусть $X_i$ распределена с плотностью $f_1(y)$ с параметрами $\lambda=1$ и $b=6$, тогда матожидание $a_1=1/\lambda+b=7$ и дисперсия $\sigma_1^2=1/\lambda^2=1$;\\\\
	применим ЦПТ:\\\\
	$\lim_{n\rightarrow+\infty}P(\sqrt{n}\frac{\oset[-.3ex]{-}{X}-a_1}{\sigma_1}<y)=\Phi(y)$\\
	$\lim_{n\rightarrow+\infty}P(\sqrt{n}(\oset[-.3ex]{-}{X}-7)<y)=\Phi(y)$\\
	так как экспоненциальное распределение абсолютно непрерывно, то можно перейти к нестрогому неравенству\\\\
	$\lim_{n\rightarrow+\infty}P(\sqrt{n}(\oset[-.3ex]{-}{X}-7)\le y)=\Phi(y)$;\\\\
	$
	\alpha_1(\delta_n)=P_{H_0}(\oset[-.3ex]{-}{X}\le 3.5+1/\sqrt{n})\\\\
	\oset[-.3ex]{-}{X}\le 3.5+1/\sqrt{n}\\\\
	\sqrt{n}\oset[-.3ex]{-}{X}\le 3.5\sqrt{n}+1\\\\
	\sqrt{n}(\oset[-.3ex]{-}{X}-7)\le -3.5\sqrt{n}+1\\\\
	\lim_{n\rightarrow+\infty}\alpha_1(\delta_n)=\lim_{n\rightarrow+\infty}P(\sqrt{n}(\oset[-.3ex]{-}{X}-7)\le -3.5\sqrt{n}+1)=\\\\
	\lim_{n\rightarrow+\infty}\Phi(-3.5\sqrt{n}+1)=\Phi(-\infty)=0
	$;\\\\
	пусть $X_i$ распределена с плотностью $f_2(y)$ с параметрами $\lambda=2$ и $b=3$, тогда матожидание $a_2=1/\lambda+b=3.5$ и дисперсия $\sigma_2^2=1/\lambda^2=0.25$;\\\\
	применим ЦПТ:\\\\
	$\lim_{n\rightarrow+\infty}P(\sqrt{n}\frac{\oset[-.3ex]{-}{X}-a_2}{\sigma_2}<y)=\Phi(y)$\\
	$\lim_{n\rightarrow+\infty}P(2\sqrt{n}(\oset[-.3ex]{-}{X}-3.5)<y)=\Phi(y)$\\
	так как экспоненциальное распределение абсолютно непрерывно, то можно перейти к нестрогому неравенству\\\\
	$\lim_{n\rightarrow+\infty}P(2\sqrt{n}(\oset[-.3ex]{-}{X}-3.5)\le y)=\Phi(y)$;\\\\
	$
	\oset[-.3ex]{-}{X}\le 3.5+1/\sqrt{n}\\\\
	\sqrt{n}\oset[-.3ex]{-}{X}\le 3.5\sqrt{n}+1\\\\
	2\sqrt{n}(\oset[-.3ex]{-}{X}-3.5)\le 2\\\\
	\alpha_2(\delta_n)=P_{H_1}(\oset[-.3ex]{-}{X}>3.5+1/\sqrt{n})=1-P_{H_1}(\oset[-.3ex]{-}{X}\le 3.5+1/\sqrt{n})\\\\
	\lim_{n\rightarrow+\infty}\alpha_2(\delta_n)=1-\lim_{n\rightarrow+\infty}P(2\sqrt{n}(\oset[-.3ex]{-}{X}-3.5)\le 2)=\\\\
	1-\Phi(2)=0.0228
	$.\\\\
	\subsection*{пункт b}
	выборка $\bm{X}=(X_1,...,X_n)$;\\\\
	критерий
	$
	\delta_n=\delta_n(\bm{X})=\left\{ {\begin{array}{l}
		H_0, \oset[-.3ex]{-}{X}>3.5+1/n \\
		H_1, \oset[-.3ex]{-}{X}\le 3.5+1/n \\
		\end{array} } \right.
	$\\\\\\
	ошибка первого рода $\alpha_1(\delta_n)$\\
	ошибка второго рода $\alpha_2(\delta_n)$\\\\
	пусть $X_i$ распределена с плотностью $f_1(y)$ с параметрами $\lambda=1$ и $b=6$, тогда матожидание $a_1=1/\lambda+b=7$ и дисперсия $\sigma_1^2=1/\lambda^2=1$;\\\\
	применим ЦПТ:\\\\
	$\lim_{n\rightarrow+\infty}P(\sqrt{n}\frac{\oset[-.3ex]{-}{X}-a_1}{\sigma_1}<y)=\Phi(y)$\\
	$\lim_{n\rightarrow+\infty}P(\sqrt{n}(\oset[-.3ex]{-}{X}-7)<y)=\Phi(y)$\\
	так как экспоненциальное распределение абсолютно непрерывно, то можно перейти к нестрогому неравенству\\\\
	$\lim_{n\rightarrow+\infty}P(\sqrt{n}(\oset[-.3ex]{-}{X}-7)\le y)=\Phi(y)$;\\\\
	$
	\alpha_1(\delta_n)=P_{H_0}(\oset[-.3ex]{-}{X}\le 3.5+1/n)\\\\
	\oset[-.3ex]{-}{X}\le 3.5+1/n\\\\
	\sqrt{n}\oset[-.3ex]{-}{X}\le 3.5\sqrt{n}+1/\sqrt{n}\\\\
	\sqrt{n}(\oset[-.3ex]{-}{X}-7)\le -3.5\sqrt{n}+1/\sqrt{n}\\\\
	\lim_{n\rightarrow+\infty}\alpha_1(\delta_n)=\lim_{n\rightarrow+\infty}P(\sqrt{n}(\oset[-.3ex]{-}{X}-7)\le -3.5\sqrt{n}+1/\sqrt{n})=\\\\
	\lim_{n\rightarrow+\infty}\Phi(-3.5\sqrt{n}+1/\sqrt{n})=\Phi(-\infty)=0
	$;\\\\
	пусть $X_i$ распределена с плотностью $f_2(y)$ с параметрами $\lambda=2$ и $b=3$, тогда матожидание $a_2=1/\lambda+b=3.5$ и дисперсия $\sigma_2^2=1/\lambda^2=0.25$;\\\\
	применим ЦПТ:\\\\
	$\lim_{n\rightarrow+\infty}P(\sqrt{n}\frac{\oset[-.3ex]{-}{X}-a_2}{\sigma_2}<y)=\Phi(y)$\\
	$\lim_{n\rightarrow+\infty}P(2\sqrt{n}(\oset[-.3ex]{-}{X}-3.5)<y)=\Phi(y)$\\
	так как экспоненциальное распределение абсолютно непрерывно, то можно перейти к нестрогому неравенству\\\\
	$\lim_{n\rightarrow+\infty}P(2\sqrt{n}(\oset[-.3ex]{-}{X}-3.5)\le y)=\Phi(y)$;\\\\
	$
	\oset[-.3ex]{-}{X}\le 3.5+1/n\\\\
	\sqrt{n}\oset[-.3ex]{-}{X}\le 3.5\sqrt{n}+1/\sqrt{n}\\\\
	2\sqrt{n}(\oset[-.3ex]{-}{X}-3.5)\le 2/\sqrt{n}\\\\
	\alpha_2(\delta_n)=P_{H_1}(\oset[-.3ex]{-}{X}>3.5+1/n)=1-P_{H_1}(\oset[-.3ex]{-}{X}\le 3.5+1/n)\\\\
	\lim_{n\rightarrow+\infty}\alpha_2(\delta_n)=1-\lim_{n\rightarrow+\infty}P(2\sqrt{n}(\oset[-.3ex]{-}{X}-3.5)\le 2/\sqrt{n})=\\\\
	1-\Phi(0)=0.5
	$.\\\\
	\subsection*{пункт c}
	выборка $\bm{X}=(X_1,...,X_n)$;\\\\
	критерий
	$
	\delta_n=\delta_n(\bm{X})=\left\{ {\begin{array}{l}
		H_0, \oset[-.3ex]{-}{X}>3.5 \\
		H_1, \oset[-.3ex]{-}{X}\le 3.5 \\
		\end{array} } \right.
	$\\\\\\
	ошибка первого рода $\alpha_1(\delta_n)$\\
	ошибка второго рода $\alpha_2(\delta_n)$\\\\
	пусть $X_i$ распределена с плотностью $f_1(y)$ с параметрами $\lambda=1$ и $b=6$, тогда матожидание $a_1=1/\lambda+b=7$ и дисперсия $\sigma_1^2=1/\lambda^2=1$;\\\\
	применим ЦПТ:\\\\
	$\lim_{n\rightarrow+\infty}P(\sqrt{n}\frac{\oset[-.3ex]{-}{X}-a_1}{\sigma_1}<y)=\Phi(y)$\\
	$\lim_{n\rightarrow+\infty}P(\sqrt{n}(\oset[-.3ex]{-}{X}-7)<y)=\Phi(y)$\\
	так как экспоненциальное распределение абсолютно непрерывно, то можно перейти к нестрогому неравенству\\\\
	$\lim_{n\rightarrow+\infty}P(\sqrt{n}(\oset[-.3ex]{-}{X}-7)\le y)=\Phi(y)$;\\\\
	$
	\alpha_1(\delta_n)=P_{H_0}(\oset[-.3ex]{-}{X}\le 3.5)\\\\
	\oset[-.3ex]{-}{X}\le 3.5\\\\
	\sqrt{n}\oset[-.3ex]{-}{X}\le 3.5\sqrt{n}\\\\
	\sqrt{n}(\oset[-.3ex]{-}{X}-7)\le -3.5\sqrt{n}\\\\
	\lim_{n\rightarrow+\infty}\alpha_1(\delta_n)=\lim_{n\rightarrow+\infty}P(\sqrt{n}(\oset[-.3ex]{-}{X}-7)\le -3.5\sqrt{n})=\\\\
	\lim_{n\rightarrow+\infty}\Phi(-3.5\sqrt{n})=\Phi(-\infty)=0
	$;\\\\
	пусть $X_i$ распределена с плотностью $f_2(y)$ с параметрами $\lambda=2$ и $b=3$, тогда матожидание $a_2=1/\lambda+b=3.5$ и дисперсия $\sigma_2^2=1/\lambda^2=0.25$;\\\\
	применим ЦПТ:\\\\
	$\lim_{n\rightarrow+\infty}P(\sqrt{n}\frac{\oset[-.3ex]{-}{X}-a_2}{\sigma_2}<y)=\Phi(y)$\\
	$\lim_{n\rightarrow+\infty}P(2\sqrt{n}(\oset[-.3ex]{-}{X}-3.5)<y)=\Phi(y)$\\
	так как экспоненциальное распределение абсолютно непрерывно, то можно перейти к нестрогому неравенству\\\\
	$\lim_{n\rightarrow+\infty}P(2\sqrt{n}(\oset[-.3ex]{-}{X}-3.5)\le y)=\Phi(y)$;\\\\
	$
	\oset[-.3ex]{-}{X}\le 3.5\\\\
	\sqrt{n}\oset[-.3ex]{-}{X}\le 3.5\sqrt{n}\\\\
	2\sqrt{n}(\oset[-.3ex]{-}{X}-3.5)\le 0\\\\
	\alpha_2(\delta_n)=P_{H_1}(\oset[-.3ex]{-}{X}>3.5)=1-P_{H_1}(\oset[-.3ex]{-}{X}\le 3.5)\\\\
	\lim_{n\rightarrow+\infty}\alpha_2(\delta_n)=1-\lim_{n\rightarrow+\infty}P(2\sqrt{n}(\oset[-.3ex]{-}{X}-3.5)\le 0)=\\\\
	1-\Phi(0)=0.5
	$.\\\\
	\section*{2}
	выборка $\bm{X}=(X_1,...,X_n)$;\\\\
	критерий
	$
	\delta=\delta(\bm{X})=\left\{ {\begin{array}{l}
			H_0, X_i\notin \mathbb{N} \\
			H_1, X_i\in \{0\}\cup\mathbb{N} \\
	\end{array} } \right.
	$\\\\\\
	ошибка первого рода: $\alpha_1(\delta)=P_{H_0}(X_i\in\mathbb{N})=0$, так как нормальное распределение абсолютно непрерывно и вероятность получить 0 или натуральное число равна нулю;\\\\
	ошибка второго рода: $\alpha_2(\delta)=P_{H_1}(X_i\notin \mathbb{N})=0$, так как случайная величина с распределением Пуассона может быть равна 0, 1, 2, 3, ..., но не дробному числу.
	\section*{3}
	
	\end{large}
\end{document}