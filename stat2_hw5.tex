\documentclass{article}
\usepackage[utf8]{inputenc}
\usepackage{amsmath,amssymb}
\usepackage{amsfonts}
\usepackage{newpxmath,newpxtext}
\usepackage[russian]{babel}
\usepackage{bm}
\usepackage[T1]{fontenc}
\author{Борис Егоров}
\title{ДЗ5 статистика}

\makeatletter
\newcommand{\oset}[3][0ex]{%
	\mathrel{\mathop{#3}\limits^{
			\vbox to#1{\kern-2\ex@
				\hbox{$\scriptstyle#2$}\vss}}}}
\makeatother

\begin{document}
	\maketitle
	\begin{large}
	\section*{1}
	i. $H_0:\mu=600;H_1:\mu<600$\\\\
	В нулевой гипотезе равенство, потому что тогда статистика гипотезы\\\\
	$t=\sqrt{n}\frac{\oset[-.3ex]{-}{X}-\mu}{\sqrt{S_0^2}}$,\\\\
	где $\oset[-.3ex]{-}{X}$ - выборочное среднее, $S_0^2$ - исправленная выборочная дисперсия.\\\\
	Если взять нулевую гипотезу $\mu \ge 600$, то непонятно, какая будет статистика гипотезы.\\\\
	В альтернативной гипотезе $\mu<600$, так как это утверждение менеджера, которое нужно проверить. Если представлено достаточно доказательств, что прибыль упала, то отвергаем нулевую гипотезу и считаем, что средний чек за выходные < 600\$. Если не смогли отвергнуть нулевую гипотезу, то считаем, что средний чек за выходные = 600\$.
	\section*{2}
	Обозначим:\\
	$\mu$ - часть акций, которые выросли;\\
	$X_i$- акция из выборки;\\
	$X_i=1$, если акция выросла;\\
	$X_i=0$, если акция не выросла;\\
	$n=50$ - объем выборки;\\
	$\oset[-.3ex]{-}{X}=24/50=0.48$ - выборочное среднее;\\
	$\alpha=0.01$ - уровень значимости.\\
	Тогда нулевая гипотеза $H_0:\mu=0.3$, альтернативная гипотеза $H_1:\mu\ne0.3$.\\
	Тогда $P_{H_0}(\oset[-.3ex]{-}{X}=0.48)=\binom{50}{24}0.3^{24}0.7^{26}\approx0.003<1-\alpha$,\\
	поэтому отклоняем нулевую гипотезу и принимаем альтернативную.
	\section*{3}
	\subsection*{a.}
	Пусть $X_1,...,X_i,...,X_n$ - выборка.\\
	Применим ЦПТ:\\\\
	$\sqrt{n}\frac{\oset[-.3ex]{-}{X}-\mu}{\sigma}=\sqrt{n}\frac{\oset[-.3ex]{-}{X}-120}{5}\sim N(0,1)$;\\\\
	статистика критерия $z=\sqrt{n}\frac{\oset[-.3ex]{-}{X}-120}{5}$.\\\\
	Так как уровень значимости $\alpha=0.05$, то $H_0$ не будет отвергнута при $-1.96<z<1.96$.\\\\
	Обозначим матожидание $X_i$ как $\mu_1$.\\\\
	Так как $H_0$ должна быть отвергнута с вероятностью 0.98 при $\mu_1=117$, то она не будет отвергнута с вероятностью 1 - 0.98 = 0.02.\\\\
	Тогда:\\\\
	$
	P(\mu1=117\cap-1.96<z<1.96)=0.02\\\\
	P(\mu1=117\cap-1.96<\sqrt{n}\frac{\oset[-.3ex]{-}{X}-\mu}{\sigma}<1.96)=0.02\\\\
	P(\mu1=117\cap\mu-\frac{1.96\sigma}{\sqrt{n}}<\oset[-.3ex]{-}{X}<\mu+\frac{1.96\sigma}{\sqrt{n}})=0.02\\\\
	P(\sqrt{n}\frac{\mu-\frac{1.96\sigma}{\sqrt{n}}-\mu_1}{\sigma}<\sqrt{n}\frac{\oset[-.3ex]{-}{X}-\mu_1}{\sigma}<\sqrt{n}\frac{\mu+\frac{1.96\sigma}{\sqrt{n}}-\mu_1}{\sigma})=0.02\\\\
	P(\sqrt{n}\frac{\mu-\mu_1}{\sigma}-1.96<\sqrt{n}\frac{\oset[-.3ex]{-}{X}-\mu_1}{\sigma}<\sqrt{n}\frac{\mu-\mu_1}{\sigma}+1.96)=0.02\\\\
	\Phi(\sqrt{n}\frac{\mu-\mu_1}{\sigma}+1.96)-\Phi(\sqrt{n}\frac{\mu-\mu_1}{\sigma}-1.96)=0.02\\\\
	\Phi(0.6\sqrt{n}+1.96)-\Phi(0.6\sqrt{n}-1.96)=0.02\\\\
	$;
	при $n>4$ $\Phi(0.6\sqrt{n}+1.96)\approx1$, поэтому можно считать $\Phi(0.6\sqrt{n}-1.96)=0.98$:\\\\
	$
	0.6\sqrt{n}-1.96=2.06\\\\
	\sqrt{n}=6.7\\\\
	n=45
	$
	\subsection*{b.}
	Ошибка второго рода (с учетом найденного в пункте a.):\\\\
	$\alpha_2(\mu_1)=\Phi(\sqrt{n}\frac{\mu-\mu_1}{\sigma}+1.96)-\Phi(\sqrt{n}\frac{\mu-\mu_1}{\sigma}-1.96)$\\\\
	Так как плотность нормального распределения симметрична, то\\\\
	$
	\alpha_2(\mu_1)=\Phi(\sqrt{n}\frac{|\mu-\mu_1|}{\sigma}+1.96)-\Phi(\sqrt{n}\frac{|\mu-\mu_1|}{\sigma}-1.96)=\\\\
	\Phi(\sqrt{45}\frac{|120-\mu_1|}{5}+1.96)-\Phi(\sqrt{45}\frac{|120-\mu_1|}{5}-1.96)
	$\\\\
	$\alpha_2(117)=\alpha_2(123)=0.02$ (вычислено в пункте a.)\\\\
	$\alpha_2(118)=\alpha_2(122)=\Phi(\sqrt{45}\frac{|120-118|}{5}+1.96)-\Phi(\sqrt{45}\frac{|120-118|}{5}-1.96)=\Phi(4.64)-\Phi(0.72)=0.2358$\\\\
	$\alpha_2(119)=\alpha_2(121)=\Phi(\sqrt{45}\frac{|120-198|}{5}+1.96)-\Phi(\sqrt{45}\frac{|120-119|}{5}-1.96)=\Phi(3.30)-\Phi(-0.62)=0.9995-(1-0.7324)=0.7319$\\\\
	\section*{4}
	Обозначим:\\
	$\mu=900$ средняя оценка за все предыдущие годы;\\
	$\sigma=180$ корень из дисперсии;\\
	$H_0:\mu=900$ нулевая гипотеза о том, что средняя оценка в этом году не изменилась;\\
	$H_1:\mu\ne900$ альтернативная гипотеза о том, что средняя оценка в этом году изменилась;\\
	$n=200$ объем выборки;\\
	$\oset[-.3ex]{-}{X}=935$ выборочное среднее;\\
	$\alpha=0.05$ уровень значимости.
	Применим ЦПТ:\\\\
	$
	z=\sqrt{n}\frac{\oset[-.3ex]{-}{X}-\mu}{\sigma}=\sqrt{200}\frac{935-900}{180}=2.75>1.96\\\\
	$
	отвергаем $H_0$.
	\section*{5}
	Обозначим:\\
	$\alpha=0.02$ уровень значимости;\\
	$\mu=15$ ожидаемый средний доход;\\
	$\sigma=4$ корень из дисперсии;\\
	$n$ объем выборки;\\
	$\mu_1=14$ реальный средний доход;\\
	$\alpha_2=0.05$ ошибка 2 рода;\\
	$H_0:\mu=15$ нулевая гипотеза;\\
	$H_1:\mu<15$ альтернативная гипотеза.\\
	Так как дисперсия известна, то применим ЦПТ:\\\\
	$
	P(-z<\sqrt{n}\frac{\oset[-.3ex]{-}{X}-\mu}{\sigma}<z)=1-\alpha\\\\
	\Phi(z)-\Phi(-z)=1-\alpha\\\\
	2\Phi(z)-1=1-\alpha\\\\
	\Phi(z)=\frac{2-\alpha}{2}=0.99\\\\
	z=2.33
	$.\\\\
	Задача аналогична задаче 3:\\\\
	$
	\Phi(\sqrt{n}\frac{|\mu-\mu_1|}{\sigma}+z)-\Phi(\sqrt{n}\frac{|\mu-\mu_1|}{\sigma}-z)=\alpha_2\\\\
	\Phi(0.25\sqrt{n}+2.33)-\Phi(0.25\sqrt{n}-2.33)=0.05\\\\
	$
	при $n>20$ $\Phi(0.25\sqrt{n}+2.33)\approx1$, поэтому можно считать $\Phi(0.25\sqrt{n}-2.33)=0.95$:\\\\
	$
	0.25\sqrt{n}-2.33=1.96\\\\
	\sqrt{n}=17.16\\\\
	$
	тогда $\alpha_2<0.05$ при $n\ge295$;\\
	ответ 295.
	\end{large}
\end{document}