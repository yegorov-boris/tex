\documentclass[a4paper]{article}
\usepackage[utf8]{inputenc}
\usepackage[russian]{babel}
\usepackage[T1]{fontenc}
\usepackage{amsmath}
\usepackage{amsfonts}
\usepackage{amssymb}
\usepackage{graphicx}
\author{Борис Егоров}
\title{ДЗ3 статистика}
\begin{document}
	\maketitle
	\section*{задание 1.1}
	\begin{large}
		Так как натуральные числа различны, то $X=i$ и $X=j$ - несовместные события для $i \ne j$,\\
		\\
		тогда $P(X \in \lbrace i,j \rbrace)=P(X=i)+P(X=j)$,\\
		\\
		тогда $1=P(X \in \mathbb{N})=\sum_{i=1}^{\infty}P(X=i) = k\sum_{i=1}^{\infty}\frac{1}{i^2}$ и $\sum_{i=1}^{\infty}\frac{1}{i^2}=\frac{1}{k}$.\\
		\\
		Обозначим как $p$ вероятность того, что $X$ нечетно:\\
		\\
		$p=k(1+\frac{1}{3^2}+\frac{1}{5^2}+\dots)=\\
		\\
		1-k(\frac{1}{2^2}+\frac{1}{4^2}+\frac{1}{6^2}+\dots)=\\
		\\
		1-\frac{k}{4}(1+\frac{1}{3^2}+\frac{1}{5^2}+\dots)=
		1-\frac{k}{4}\sum_{i=1}^{\infty}\frac{1}{i^2}=1-\frac{k}{4} \cdot \frac{1}{k}=1-\frac{1}{4}=\frac{3}{4}$\\
	\end{large}
	\section*{задание 1.2}
	\begin{large}
		\subsection*{1}
		$1=\int_{0}^{2}p(x)dx=\int_{0}^{2}C(1-|x-1|)dx=C\int_{0}^{2}(1-|x-1|)dx=\\
		\\
		C(\int_{0}^{1}(1-(1-x))dx+\int_{1}^{2}(1-(x-1))dx)=\\
		\\
		C(\int_{0}^{1}xdx+\int_{1}^{2}(2-x)dx)=C(\frac{1}{2}+4-2-(\frac{4}{2}-\frac{1}{2}))=C$\\
		\\
		$C=1$
		\subsection*{2}
		для $0\le x<1$ получим $1-|x-1|=x$ и $p(x)=x$, тогда\\
		\\
		$F(x)=\int_{0}^{x}p(x)dx=\int_{0}^{x}xdx=\frac{x^2}{2}$;\\
		\\
		для $1 \le x < 2$ получим $1-|x-1|=2-x$ и $p(x)=2-x$, тогда\\
		\\
		$F(x)=\int_{0}^{x}p(x)dx=\frac{1}{2}+\int_{1}^{x}p(x)dx=\frac{1}{2}+\int_{1}^{x}(2-x)dx=\\
		\\
		\frac{1}{2}+2x-2-\frac{x^2}{2}+\frac{1}{2}=-\frac{x^2}{2}+2x-1$;\\
		\\
		\[ F(x) = \left\{ \begin{array}{ll}
			0 & \mbox{, $x<0$};\\
			\\
			\frac{x^2}{2} & \mbox{, $0\le x<1$};\\
			\\
			-\frac{x^2}{2}+2x-1 & \mbox{, $1 \le x < 2$};\\
			\\
			1 & \mbox{, $x \ge 2$}.\end{array} \right. \]
		\subsection*{3}
		можно проверить, что $p(x)$ непрерывна на $[0,2]$, поэтому\\
		\\
		$P(X \in [-1,1])=F(1)-F(-1)=\frac{1}{2}-0=\frac{1}{2}$
		\subsection*{4}
		$X \in [0,2] \Rightarrow Y=X^2 \in [0, 4] \Rightarrow \sqrt{Y}=X$;\\
		\\
		$F_Y(y)=P(Y<y)=P(X^2<y)=P(X< \sqrt{y})=F_X(\sqrt{y})$;\\
		\\
		пусть $0 \le \sqrt{y}<1$, тогда $0 \le y < 1$,\\
		\\
		получим $F_Y(y)=F_X(\sqrt{y})=\frac{(\sqrt{y})^2}{2}=\frac{y}{2}$ - абсолютно непрерывна на $[0,1)$,\\
		\\
		поэтому имеет плотность $p_Y(y)=\frac{dF_Y(y)}{dy}=\frac{d\frac{y}{2}}{dy}=\frac{1}{2}$ на $[0,1)$;\\
		\\
		пусть $1 \le y \le 4$, тогда $1 \le x \le 2$,\\
		\\
		получим $F_Y(y)=F_X(\sqrt{y})=-\frac{(\sqrt{y})^2}{2}+2\sqrt{y}-1=-\frac{y}{2}+2\sqrt{y}-1$ - абсолютно непрерывна на $[1,4]$,\\
		\\
		поэтому имеет плотность $p_Y(y)=\frac{dF_Y(y)}{dy}=\frac{d(-\frac{y}{2}+2\sqrt{y}-1)}{dy}=\frac{1}{\sqrt{y}}-\frac{1}{2}$ на $[1,4]$;\\
		\\
		\[ p(y) = \left\{ \begin{array}{ll}
			0 & \mbox{, $y<0$};\\
			\\
			\frac{1}{2} & \mbox{, $0\le y<1$};\\
			\\
			\frac{1}{\sqrt{y}}-\frac{1}{2} & \mbox{, $1 \le y < 4$};\\
			\\
			1 & \mbox{, $y \ge 4$}.\end{array} \right. \]
	\end{large}
\end{document}