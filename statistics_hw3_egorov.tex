\documentclass[a4paper]{article}
\usepackage[utf8]{inputenc}
\usepackage[russian]{babel}
\usepackage[T1]{fontenc}
\usepackage{amsmath}
\usepackage{amsfonts}
\usepackage{amssymb}
\usepackage{graphicx}
\author{Борис Егоров}
\title{ДЗ3 статистика}
\begin{document}
	\maketitle
	\section*{задание 1.1}
	\begin{large}
		Так как натуральные числа различны, то $X=i$ и $X=j$ - несовместные события для $i \ne j$,\\
		\\
		тогда $P(X \in \lbrace i,j \rbrace)=P(X=i)+P(X=j)$,\\
		\\
		тогда $1=P(X \in \mathbb{N})=\sum_{i=1}^{\infty}P(X=i) = k\sum_{i=1}^{\infty}\frac{1}{i^2}$ и $\sum_{i=1}^{\infty}\frac{1}{i^2}=\frac{1}{k}$.\\
		\\
		Обозначим как $p$ вероятность того, что $X$ нечетно:\\
		\\
		$p=k(1+\frac{1}{3^2}+\frac{1}{5^2}+\dots)=\\
		\\
		1-k(\frac{1}{2^2}+\frac{1}{4^2}+\frac{1}{6^2}+\dots)=\\
		\\
		1-\frac{k}{4}(1+\frac{1}{3^2}+\frac{1}{5^2}+\dots)=
		1-\frac{k}{4}\sum_{i=1}^{\infty}\frac{1}{i^2}=1-\frac{k}{4} \cdot \frac{1}{k}=1-\frac{1}{4}=\frac{3}{4}$\\
	\end{large}
	\section*{задание 1.2}
	\begin{large}
		\subsection*{1}
		$1=\int_{0}^{2}p(x)dx=\int_{0}^{2}C(1-|x-1|)dx=C\int_{0}^{2}(1-|x-1|)dx=\\
		\\
		C(\int_{0}^{1}(1-(1-x))dx+\int_{1}^{2}(1-(x-1))dx)=\\
		\\
		C(\int_{0}^{1}xdx+\int_{1}^{2}(2-x)dx)=C(\frac{1}{2}+4-2-(\frac{4}{2}-\frac{1}{2}))=C$\\
		\\
		$C=1$
		\subsection*{2}
		для $0\le x<1$ получим $1-|x-1|=x$ и $p(x)=x$, тогда\\
		\\
		$F(x)=\int_{0}^{x}p(x)dx=\int_{0}^{x}xdx=\frac{x^2}{2}$;\\
		\\
		для $1 \le x < 2$ получим $1-|x-1|=2-x$ и $p(x)=2-x$, тогда\\
		\\
		$F(x)=\int_{0}^{x}p(x)dx=\frac{1}{2}+\int_{1}^{x}p(x)dx=\frac{1}{2}+\int_{1}^{x}(2-x)dx=\\
		\\
		\frac{1}{2}+2x-2-\frac{x^2}{2}+\frac{1}{2}=-\frac{x^2}{2}+2x-1$;\\
		\\
		\[ F(x) = \left\{ \begin{array}{ll}
			0 & \mbox{, $x<0$};\\
			\\
			\frac{x^2}{2} & \mbox{, $0\le x<1$};\\
			\\
			-\frac{x^2}{2}+2x-1 & \mbox{, $1 \le x < 2$};\\
			\\
			1 & \mbox{, $x \ge 2$}.\end{array} \right. \]
		\subsection*{3}
		можно проверить, что $p(x)$ непрерывна на $[0,2]$, поэтому\\
		\\
		$P(X \in [-1,1])=F(1)-F(-1)=\frac{1}{2}-0=\frac{1}{2}$
		\subsection*{4}
		$X \in [0,2] \Rightarrow Y=X^2 \in [0, 4] \Rightarrow \sqrt{Y}=X$;\\
		\\
		$F_Y(y)=P(Y<y)=P(X^2<y)=P(X< \sqrt{y})=F_X(\sqrt{y})$;\\
		\\
		пусть $0 \le \sqrt{y}<1$, тогда $0 \le y < 1$,\\
		\\
		получим $F_Y(y)=F_X(\sqrt{y})=\frac{(\sqrt{y})^2}{2}=\frac{y}{2}$ - абсолютно непрерывна на $[0,1)$,\\
		\\
		поэтому имеет плотность $p_Y(y)=\frac{dF_Y(y)}{dy}=\frac{d\frac{y}{2}}{dy}=\frac{1}{2}$ на $[0,1)$;\\
		\\
		пусть $1 \le y \le 4$, тогда $1 \le x \le 2$,\\
		\\
		получим $F_Y(y)=F_X(\sqrt{y})=-\frac{(\sqrt{y})^2}{2}+2\sqrt{y}-1=-\frac{y}{2}+2\sqrt{y}-1$ - абсолютно непрерывна на $[1,4]$,\\
		\\
		поэтому имеет плотность $p_Y(y)=\frac{dF_Y(y)}{dy}=\frac{d(-\frac{y}{2}+2\sqrt{y}-1)}{dy}=\frac{1}{\sqrt{y}}-\frac{1}{2}$ на $[1,4]$;\\
		\[ p(y) = \left\{ \begin{array}{ll}
			0 & \mbox{, $y<0$};\\
			\\
			\frac{1}{2} & \mbox{, $0\le y<1$};\\
			\\
			\frac{1}{\sqrt{y}}-\frac{1}{2} & \mbox{, $1 \le y < 4$};\\
			\\
			1 & \mbox{, $y \ge 4$}.\end{array} \right. \]
	\end{large}
	\section*{задание 1.3}
	\begin{large}
		\subsection*{1}
		$X \in [0,1] \Rightarrow Y=-logX \in [0,+\infty)$;\\
		\\
		$F(y)=P(Y<y)=P(-logX<y)=P(log\frac{1}{X}<y)=\\
		\\
		P(\frac{1}{X}<e^y)=P(X>e^{-y})=1-e^{-y}$;\\
		\\
		$p(y)=\frac{dF(y)}{dy}=\frac{d(1-e^{-y})}{dy}=e^{-y}$\\
		\[ p(y) = \left\{ \begin{array}{ll}
			0 & \mbox{, $y<0$};\\
			\\
			e^{-y} & \mbox{, $y \ge 0$}.\end{array} \right. \]
		\subsection*{2}
		$X \in [0,1] \Rightarrow Y=2X+1 \in [1,3]$;\\
		\\
		$F(y)=P(Y<y)=P(2X+1<y)=P(2X<y-1)=\\
		\\
		P(X<\frac{y-1}{2})=\frac{y-1}{2}$;\\
		\\
		$p(y)=\frac{dF(y)}{dy}=\frac{d\frac{y-1}{2}}{dy}=\frac{1}{2}$\\
		\[ p(y) = \left\{ \begin{array}{ll}
			0 & \mbox{, $y \notin [1,3]$};\\
			\\
			\frac{1}{2} & \mbox{, $y \in [1,3]$}.\end{array} \right. \]
		\subsection*{3}
		$X \in [0,1] \Rightarrow Y=X-\frac{1}{X} \in (-\infty,0])$;\\
		\\
		$F(y)=P(Y<y)=P(X-\frac{1}{X}<y)$;\\
		\\
		$
		X-\frac{1}{X}<y\\\\
		X^2-1<Xy\\\\
		X^2-yX-1<0\\\\
		(X-\frac{y}{2})^2-\frac{y^2}{4}-1<0\\\\
		(X-\frac{y}{2})^2<1+\frac{y^2}{4}\\\\
		|X-\frac{y}{2}|<\sqrt{1+\frac{y^2}{4}}\\\\
		\left. \begin{array}{rr}
			X\in [0,1]\\
			Y\in (-\infty,0]\end{array}\right\} \Rightarrow X-\frac{y}{2}\ge0\\\\
		X-\frac{y}{2}<\sqrt{1+\frac{y^2}{4}}\\\\
		X<\frac{y}{2}+\sqrt{1+\frac{y^2}{4}}\\\\
		F(y)=P(X<\frac{y}{2}+\sqrt{1+\frac{y^2}{4}})=F_X(\frac{y}{2}+\sqrt{1+\frac{y^2}{4}})=\frac{y}{2}+\sqrt{1+\frac{y^2}{4}}\\\\
		p(y)=\frac{dF(y)}{dy}=\frac{d(\frac{y}{2}+\sqrt{1+\frac{y^2}{4}})}{dy}=\frac{1}{2}+\frac{1}{2\sqrt{1+\frac{y^2}{4}}} \cdot \frac{1}{4} \cdot 2y=\frac{1}{2}+\frac{y}{4\sqrt{1+\frac{y^2}{4}}}\\\\\\
		p(y) = \left\{ \begin{array}{ll}
			0 & \mbox{, $y>0$};\\
			\\
			\frac{1}{2}+\frac{y}{4\sqrt{1+\frac{y^2}{4}}} & \mbox{, $y \le 0$}.\end{array} \right.
		$
		\subsection*{4}
		$X \in [0,1] \Rightarrow Y=-log(1-X) \in [0,+\infty)$;\\\\
		$F(y)=P(Y<y)=P(-log(1-X)<y)$;\\\\
		$
		-log(1-X)<y\\\\
		log\frac{1}{1-X}<y\\\\
		\frac{1}{1-X}<e^y\\\\
		1-X>e^{-y}\\\\
		-X>e^{-y}-1\\\\
		X<1-e^{-y}\\\\
		p(y)=\frac{dF(y)}{dy}=\frac{d(1-e^{-y})}{dy}=e^{-y}\\\\\\
		p(y) = \left\{ \begin{array}{ll}
			0 & \mbox{, $y<0$};\\
			\\
			e^{-y} & \mbox{, $y \ge 0$}.\end{array} \right.
		$
		\subsection*{5}
		$
		X \in [0,1] \Rightarrow Y=X^2 \in [0,1];\\\\
		F(y)=P(Y<y)=P(X^2<y)=P(X<\sqrt{y})=F_X(\sqrt{y})=\sqrt{y};\\\\
		p(y)=\frac{dF(y)}{dy}=\frac{d(\sqrt{y})}{dy}=\frac{1}{2\sqrt{y}}\\\\\\
		p(y) = \left\{ \begin{array}{ll}
			0 & \mbox{, $y \notin [0,1]$};\\
			\\
			\frac{1}{2\sqrt{y}} & \mbox{, $y \in [0,1]$}.\end{array} \right.
		$
		\subsection*{6}
		$
		X \in [0,1] \Rightarrow Y=e^{X-1}\in [e^{-1},1];\\\\
		F(y)=P(Y<y)=P(e^{X-1}<y)=P(X-1<logy)=\\\\
		P(X<1+logy)=F_X(1+logy)=1+logy;\\\\
		p(y)=\frac{dF(y)}{dy}=\frac{d(1+logy)}{dy}=\frac{1}{y}\\\\\\
		p(y) = \left\{ \begin{array}{ll}
			0 & \mbox{, $y \notin [e^{-1},1]$};\\
			\\
			\frac{1}{y} & \mbox{, $y \in [e^{-1},1]$}.\end{array} \right.
		$
	\section*{задание 1.6}
	$F(X)\in[0,1]$ по определению;\\\\
	так как $X\in[0,1]$, а $F(X)$ непрерывна и не убывает, то точная верхняя грань $sup\{t|F(t)=x\}$ достигается в любой точке $x\in[0,1]$, поэтому\\\\
	$P(F(x)<x)=P(X<sup\{t|F(t)=x\})=\\\\F(sup\{t|F(t)=x\})=x$
	\section*{задание 1.8}
	$
	\frac{X}{|X|}=\left\{ \begin{array}{ll}
		1 &  \mbox{, $X>0$}\\
		-1 & \mbox{, $X<0$}\end{array}\right.
	$\\\\
	так как $F(X)$ непрерывна в $0$, то $P(X\ge 0)=1-F(0)$;\\\\
	тогда:\\\\
	$
	Y\in\{-1,1\}\\
	P(Y=-1)=P(X<0)=F(0)\\
	P(Y=1)=P(X=0)+P(X>0)=P(X\ge0)=1-F(0)\\
	$
	\section*{задание 1.9}
	пусть $X$ - диаметр круга, тогда площадь круга $S(X)=\frac{\pi X^2}{4}$;\\\\
	выразим диаметр через площадь: $X(S)=2\sqrt{\frac{S}{\pi}}$;\\\\
	тогда функция распределения площади\\\\ $F_S(s)=P(S<s)=P(X<2\sqrt{\frac{s}{\pi}})=2\sqrt{\frac{s}{\pi}}-a$,\\\\
	тогда плотность распределения $p_S(s)=\frac{dF_S(s)}{ds}=\frac{d(2\sqrt{\frac{s}{\pi}}-a)}{ds}=\\\\2\cdot \frac{1}{2}\cdot \frac{1}{\sqrt{\frac{s}{\pi}}}\cdot \frac{1}{\pi}=\frac{1}{\pi\sqrt{\frac{s}{\pi}}}=\frac{1}{\sqrt{\frac{\pi^2s}{\pi}}}=\frac{1}{\sqrt{\pi s}}$;\\\\
	распределение площади для множества $B\subset[\frac{\pi a^2}{4},\frac{\pi b^2}{4}]$:\\\\
	$P(S\in B)=\frac{1}{\sqrt{\pi}}\int_B \frac{ds}{\sqrt{s}}$
	\section*{задание 1.10}
	
	\end{large}
\end{document}