\documentclass[a4paper]{article}
\usepackage[utf8]{inputenc}
\usepackage[russian]{babel}
\usepackage[T1]{fontenc}
\usepackage{amsmath}
\usepackage{amsfonts}
\usepackage{amssymb}
\usepackage{graphicx}
\author{Борис Егоров}
\title{ДЗ3 статистика}
\begin{document}
	\maketitle
	\section*{задание 1.1}
	\begin{large}
		Так как натуральные числа различны, то $X=i$ и $X=j$ - несовместные события для $i \ne j$,\\
		\\
		тогда $P(X \in \lbrace i,j \rbrace)=P(X=i)+P(X=j)$,\\
		\\
		тогда $1=P(X \in \mathbb{N})=\sum_{i=1}^{\infty}P(X=i) = k\sum_{i=1}^{\infty}\frac{1}{i^2}$ и $\sum_{i=1}^{\infty}\frac{1}{i^2}=\frac{1}{k}$.\\
		\\
		Обозначим как $p$ вероятность того, что $X$ нечетно:\\
		\\
		$p=k(1+\frac{1}{3^2}+\frac{1}{5^2}+\dots)=\\
		\\
		1-k(\frac{1}{2^2}+\frac{1}{4^2}+\frac{1}{6^2}+\dots)=\\
		\\
		1-\frac{k}{4}(1+\frac{1}{3^2}+\frac{1}{5^2}+\dots)=
		1-\frac{k}{4}\sum_{i=1}^{\infty}\frac{1}{i^2}=1-\frac{k}{4} \cdot \frac{1}{k}=1-\frac{1}{4}=\frac{3}{4}$\\
	\end{large}
\end{document}