\documentclass[a4paper]{article}
\usepackage[utf8]{inputenc}
\usepackage[russian]{babel}
\usepackage[T1]{fontenc}
\usepackage{amsmath}
\usepackage{delarray}
\usepackage{amsfonts}
\usepackage{amssymb}
\usepackage{graphicx}
\author{Борис Егоров}
\title{ДЗ7 статистика}
\begin{document}
	\maketitle
	\begin{large}
	\section*{задание 1.1}
	$
	P=\left(
	\begin{array}{ll}
		1-a & a\\
		b & 1-b
	\end{array}
	\right)
	$\\\\
	так как умножение матриц ассоциативно, то\\\\
	$
	P^n=P^{n-1}P;\\\\
	\left(
		\begin{array}{ll}
		p_{00}^{(n)} & p_{01}^{(n)}\\\\
		p_{10}^{(n)} & p_{11}^{(n)}
		\end{array}
	\right)=
	\left(
	\begin{array}{ll}
		p_{00}^{(n-1)} & p_{01}^{(n-1)}\\\\
		p_{10}^{(n-1)} & p_{11}^{(n-1)}
	\end{array}
	\right)\cdot
	\left(
		\begin{array}{ll}
		1-a & a\\\\
		b & 1-b
		\end{array}
	\right)
	$,\\\\\\
	тогда $p_{00}^{(n)}=p_{00}^{(n-1)}(1-a)+p_{01}^{(n-1)}b$;\\\\
	так как матрица $P^{n-1}$ стохастическая, то $p_{01}^{(n-1)}=1-p_{00}^{(n-1)}$,\\\\
	поэтому	$p_{00}^{(n)}=p_{00}^{(n-1)}(1-a)+(1-p_{00}^{(n-1)})b=b+(1-a-b)p_{00}^{(n-1)}$;\\\\
	решим рекуррентное соотношение:\\\\
	$
	p_{00}^{(n)}=A+B(1-a-b)^n;\\\\
	n=0: A+B=1;\\\\
	n=1: A+B(1-a-b)=1-a;\\\\
	1-B+B(1-a-b)=1-a;\\\\
	1+B(1-a-b-1)=1-a;\\\\
	B(a+b)=a;\\\\
	B=\frac{a}{a+b};\\\\
	A=1-\frac{a}{a+b}=\frac{b}{a+b};
	$\\\\
	получим $p_{00}^{(n)}=\frac{b}{a+b}+\frac{a}{a+b}(1-a-b)^n$ при $a+b\ne0$;\\\\
	так как $a$ и $b$ - вероятности, то $a\ge0$ и $b\ge0$, поэтому $a+b=0$ возможно только при
	$a=0$ и $b=0$;\\\\
	\begin{bf}
		пусть $a=0$ и $b=0$:\\\\
	\end{bf}
	$
	P^n=\left(
	\begin{array}{ll}
		1 & 0\\
		0 & 1
	\end{array}
	\right)
	$\\\\
	\begin{bf}
		пусть $a\ne0$ и/или $b\ne0$:\\\\
	\end{bf}
	$p_{01}^{(n)}=1-\frac{b}{a+b}-\frac{a}{a+b}(1-a-b)^n=\frac{a}{a+b}-\frac{a}{a+b}(1-a-b)^n$;\\\\
	$p_{11}^{(n)}=p_{10}^{(n-1)}a+p_{11}^{(n-1)}(1-b)$;\\\\
	так как матрица $P^{n-1}$ стохастическая, то $p_{10}^{(n-1)}=1-p_{11}^{(n-1)}$,\\\\
	поэтому	$p_{11}^{(n)}=p_{11}^{(n-1)}(1-b)+(1-p_{11}^{(n-1)})a=a+(1-a-b)p_{11}^{(n-1)}$;\\\\
	решим рекуррентное соотношение:\\\\
	$
	p_{11}^{(n)}=A+B(1-a-b)^n;\\\\
	n=0: A+B=1;\\\\
	n=1: A+B(1-a-b)=1-b;\\\\
	1-B+B(1-a-b)=1-b;\\\\
	1+B(1-a-b-1)=1-b;\\\\
	B(a+b)=b;\\\\
	B=\frac{a}{a+b};\\\\
	A=1-\frac{b}{a+b}=\frac{a}{a+b};\\\\
	p_{11}^{(n)}=\frac{a}{a+b}+\frac{b}{a+b}(1-a-b)^n;\\\\
	p_{10}^{(n)}=1-\frac{a}{a+b}-\frac{b}{a+b}(1-a-b)^n=\frac{b}{a+b}-\frac{b}{a+b}(1-a-b)^n;\\\\
	P^n=\left(
	\begin{array}{ll}
		\frac{b}{a+b}+\frac{a}{a+b}(1-a-b)^n & \frac{a}{a+b}-\frac{a}{a+b}(1-a-b)^n\\\\
		\frac{b}{a+b}-\frac{b}{a+b}(1-a-b)^n & \frac{a}{a+b}+\frac{b}{a+b}(1-a-b)^n=\frac{1}{a+b}
	\end{array}
	\right)=\\\\\\
	\frac{1}{a+b}
	\left(
	\begin{array}{ll}
		b+a(1-a-b)^n & a-a(1-a-b)^n\\\\
		b-b(1-a-b)^n & a+b(1-a-b)^n
	\end{array}
	\right)
	$;\\\\\\
	найдем $\lim_{n\rightarrow\infty}P^n$:\\\\
	пусть $a=1,b=1$, тогда\\\\
	$
	P^n=\frac{1}{2}
	\left(
	\begin{array}{ll}
		1+(-1)^n & 1-(-1)^n\\\\
		1-(-1)^n & 1+(-1)^n
	\end{array}
	\right)
	$\\\\
	и $\lim_{n\rightarrow\infty}P^n$ не существует;\\\\
	пусть $0<a+b<2$,\\\\
	тогда $-1<1-a-b<1$ и $\lim_{n\rightarrow\infty}(1-a-b)=0$, поэтому\\\\
	$
	\lim_{n\rightarrow\infty}P^n=\frac{1}{a+b}\left(
	\begin{array}{ll}
		b & a\\
		b & a
	\end{array}
	\right)
	$.
	\section*{задание 1.2}
	\subsection*{1}
	пусть случайная величина $X_k$ - это координаты $(i_k,j_k)$ точки на k-м шаге;\\
	\begin{bf}
		рассмотрим точку $(i_n,j_n)$ внутри квадрата:\\\\
	\end{bf}
	обозначим $B_{n+1}={(i_n-1,j_n),(i_n,j_n+1),(i_n+1,j_n),(i_n,j_n-1)}$ (соседи слева, сверху, справа, снизу);\\\\
	для внутренней точки квадрата $X_{k+1}\in B_{k+1} \Leftrightarrow X_k=(i_k,j_k)$, тогда\\\\
	$
	P(X_{n+1}=(i_{n+1},j_{n+1})|X_n=(i_n,j_n),...,X_0=(i_0,j_0))=\\\\
	\left.
	\begin{array}{ll}
		\frac{1}{4},X_{n+1}\in B_{n+1}\\
		0,X_{n+1}\notin B_{n+1}
	\end{array}
	\right\}=P(X_{n+1}=(i_{n+1},j_{n+1})|X_n=(i_n,j_n))
	$\\\\
	\begin{bf}
		рассмотрим точку $(i_n,j_n)$ на границе квадрата:\\
	\end{bf}
	обозначим\\\\
	$
	A_{n+1}=(i_{n+1},j_{n+1})=\left.
	\begin{array}{ll}
		(i_n,j_n+1),i_n=0,j_n<n\\
		(i_n+1,j_n),i_n<n,j_n=n\\
		(i_n,j_n-1),i_n=n,j_n>0\\
		(i_n-1,j_n),i_n>0,j_n=0
	\end{array}
	\right\}
	$\\\\
	(следующая по часовой стрелке точка на границе квадрата);\\
	для точки $X_n$ на границе квадрата $X_{n+1}\in A_{n+1} \Leftrightarrow X_n=(i_n,j_n)$, тогда\\\\
	$
	P(X_{n+1}=(i_{n+1},j_{n+1})|X_n=(i_n,j_n),...,X_0=(i_0,j_0))=\\\\
	\left.
	\begin{array}{ll}
		1,X_{n+1}\in A_{n+1}\\
		0,X_{n+1}\notin A_{n+1}
	\end{array}
	\right\}=P(X_{n+1}=(i_{n+1},j_{n+1})|X_n=(i_n,j_n))
	$\\\\
	ответ: будет, так как вероятность состояния на шаге $n+1$ зависит только от состояния на шаге $n$ и не зависит от номера шага.
	\subsection*{2}
	пусть случайная величина $X_k$ - это координаты $(i_k,j_k)$ точки на k-м шаге;\\
	\begin{bf}
		рассмотрим точку $(i_n,j_n)$ внутри квадрата:\\\\
	\end{bf}
	обозначим $B_{n+1}={(i_n-1,j_n),(i_n,j_n+1),(i_n+1,j_n),(i_n,j_n-1)}$ (соседи слева, сверху, справа, снизу);\\\\
	для внутренней точки квадрата $X_{k+1}\in B_{k+1} \Leftrightarrow X_k=(i_k,j_k)$, тогда\\\\
	$
	P(X_{n+1}=(i_{n+1},j_{n+1})|X_n=(i_n,j_n),...,X_0=(i_0,j_0))=\\\\
	\left.
	\begin{array}{ll}
		\frac{1}{4},X_{n+1}\in B_{n+1}\\
		0,X_{n+1}\notin B_{n+1}
	\end{array}
	\right\}=P(X_{n+1}=(i_{n+1},j_{n+1})|X_n=(i_n,j_n))
	$\\\\
	\begin{bf}
		рассмотрим точку $(i_n,j_n)$ на границе квадрата:\\\\
	\end{bf}
	$
	P(X_{n+1}=(i_{n+1},j_{n+1})|X_n=(i_n,j_n),...,X_0=(i_0,j_0))=\\\\
	\left.
	\begin{array}{ll}
		1,X_{n+1}=X_{n-1}\\
		0,X_{n+1}\ne X_{n-1}
	\end{array}
	\right\}=P(X_{n+1}=(i_{n+1},j_{n+1})|X_n=(i_n,j_n))
	$\\\\
	ответ: будет, так как вероятность состояния на шаге $n+1$ зависит только от состояния на шаге $n$ и не зависит от номера шага.
	\subsection*{3}
	пусть случайная величина $X_k$ - это координаты $(i_k,j_k)$ точки на k-м шаге;\\
	\begin{bf}
		рассмотрим точку $(i_n,j_n)$ внутри квадрата:\\\\
	\end{bf}
	обозначим $B_{n+1}={(i_n-1,j_n),(i_n,j_n+1),(i_n+1,j_n),(i_n,j_n-1)}$ (соседи слева, сверху, справа, снизу);\\\\
	для внутренней точки квадрата $X_{k+1}\in B_{k+1} \Leftrightarrow X_k=(i_k,j_k)$, тогда\\\\
	$
	P(X_{n+1}=(i_{n+1},j_{n+1})|X_n=(i_n,j_n),...,X_0=(i_0,j_0))=\\\\
	\left.
	\begin{array}{ll}
		\frac{1}{4},X_{n+1}\in B_{n+1}\\
		0,X_{n+1}\notin B_{n+1}
	\end{array}
	\right\}=P(X_{n+1}=(i_{n+1},j_{n+1})|X_n=(i_n,j_n))
	$\\\\
	\begin{bf}
		рассмотрим точку $X_n=(i_n,j_n)$ на границе квадрата:\\\\
	\end{bf}
	обозначим $B$ множество точек на границе квадрата;\\
	обозначим $r_{n+1}$ следующую по часовой стрелке точку от $X_n$;\\
	обозначим $l_{n+1}$ следующую против часовой стрелки точку от $X_n$;\\\\
	$
	P(X_{n+1}=r_{n+1}|X_n=(i_n,j_n),...,X_0=(i_0,j_0))=\\\\
	\left.
	\begin{array}{ll}
		1,X_n=r_n\\
		0,X_n=l_n\\
		\frac{1}{2},X_{n-1}\notin B
	\end{array}
	\right\}=\\\\
	P(X_{n+1}=r_{n+1}|X_n=(i_n,j_n),X_{n-1}=(i_{n-1},j_{n-1}))\ne\\
	P(X_{n+1}=r_{n+1}|X_n=(i_n,j_n));\\\\
	P(X_{n+1}=l_{n+1}|X_n=(i_n,j_n),...,X_0=(i_0,j_0))=\\\\
	\left.
	\begin{array}{ll}
		1,X_n=l_n\\
		0,X_n=r_n\\
		\frac{1}{2},X_{n-1}\notin B
	\end{array}
	\right\}=\\\\
	P(X_{n+1}=l_{n+1}|X_n=(i_n,j_n),X_{n-1}=(i_{n-1},j_{n-1}))\ne\\
	P(X_{n+1}=l_{n+1}|X_n=(i_n,j_n));
	$\\\\
	ответ: нет, так как вероятность состояния на шаге $n+1$ может зависеть не только от состояния на шаге $n$, но и от состояния на шаге $n-1$.
	\section*{задание 1.7}
	обозначим $Y_n$ количество различных цифр среди $X_0,X_1,...,X_n$;\\
	предположим, что последовательность $Y_0,Y_1,...,Y_n$ образует цепь Маркова;\\
	пусть $\mathbf{P}=[p_{i,j}]$ - матрица вероятностей перехода за 1 шаг;\\
	так как $X_i$ целые и $0\le X_i \le 9$, то $1\le Y_k\le10$;\\
	$P(Y_{i+1}=n|Y_i=n)=p_{n,n}=\frac{n}{10}$ (вероятность выпадения одной из $n$ уже выпавших цифр), так как выпадение любой из 10 цифр равновероятно;\\
	$P(Y_{i+1}<n|Y_i=n)=p_{j,i}=0$ при $j<i$ (количество различных выпавших цифр не может убывать);\\
	$P(Y_{i+1}>n+1|Y_i=n)=p_{i,i+t}=0$ при $t>1$ (количество различных выпавших цифр не может за 1 шаг увеличиться больше чем на 1);\\
	получили, что из состояния $1\le n<10$ возможны переходы в состояние $n$ или в состояние $n+1$, тогда $p_{n,n+1}=1-p_{n,n}=1-\frac{n}{10}$ для $1\le n<10$;\\
	получили, что $Y_{i+1}$ зависит только от $Y_i$ (и не зависит от номера шага), то есть последовательность $Y_0,Y_1,...,Y_n$ является цепью Маркова;\\
	состояние 10 существенное, так как из него невозможно выйти и оно сообщается само с собой;\\
	состояния 1,2,...,9 несущественные, так как 10 достижимо из 1,2,...,9, но 1,2,...,9 недостижимы из 10;\\
	матрица вероятностей переходов:\\\\
	$
	\mathbf{P}=\left(
	\begin{array}{cccccccccc}
		\frac{1}{10} & \frac{9}{10} & 0 & 0 & 0 & 0 & 0 & 0 & 0 & 0\\\\
		0 & \frac{2}{10} & \frac{8}{10} & 0 & 0 & 0 & 0 & 0 & 0 & 0\\\\
		0 & 0 & \frac{3}{10} & \frac{7}{10} & 0 & 0 & 0 & 0 & 0 & 0\\\\
		0 & 0 & 0 & \frac{4}{10} & \frac{6}{10} & 0 & 0 & 0 & 0 & 0\\\\
		0 & 0 & 0 & 0 & \frac{5}{10} & \frac{5}{10} & 0 & 0 & 0 & 0\\\\
		0 & 0 & 0 & 0 & 0 & \frac{6}{10} & \frac{4}{10} & 0 & 0 & 0\\\\
		0 & 0 & 0 & 0 & 0 & 0 & \frac{7}{10} & \frac{3}{10} & 0 & 0\\\\
		0 & 0 & 0 & 0 & 0 & 0 & 0 & \frac{8}{10} & \frac{2}{10} & 0\\\\
		0 & 0 & 0 & 0 & 0 & 0 & 0 & 0 & \frac{9}{10} & \frac{1}{10}\\\\
		0 & 0 & 0 & 0 & 0 & 0 & 0 & 0 & 0 & 1
	\end{array}
	\right)
	$
	\section*{задание 1.8}
	\subsection*{1}
	обозначим стационарное распределение $\pi=(\pi_1,\pi_2,\pi_3,\pi_4)$;\\\\
	$
	\pi P=\pi;\\\\
	\left.
	\begin{array}{ll}
		\frac{\pi_1}{3}+\frac{\pi_2}{2}+\frac{\pi_3}{4}=\pi_1\\\\
		\frac{\pi_1}{3}+\frac{\pi_2}{2}+\frac{\pi_3}{4}+\frac{\pi_4}{2}=\pi_2\\\\
		\frac{\pi_1}{3}=\pi_3\\\\
		\frac{\pi_3}{2}+\frac{\pi_4}{2}=\pi_4
	\end{array}
	\right\};\\\\
	\pi_3=\pi_4;\\
	\pi_1=3\pi_4;\\
	\pi_2=2(3\pi_4-\pi_4-\frac{\pi_4}{4})=\frac{7\pi_4}{2}
	$;\\\\
	так как $\pi$ - распределение, то должно выполняться свойство $\pi_1+\pi_2+\pi_3+\pi_4=1$, тогда\\\\
	$
	3\pi_4+\frac{7\pi_4}{2}+\pi_4+\pi_4=1;\\\\
	\frac{17\pi_4}{2}=1;\\\\
	\pi_4=\frac{2}{17};\\\\
	\pi=(\frac{6}{17},\frac{7}{17},\frac{2}{17},\frac{2}{17});
	$\\\\
	подставим $\pi$ в уравнение $\frac{\pi_1}{3}+\frac{\pi_2}{2}+\frac{\pi_3}{4}+\frac{\pi_4}{2}=\pi_2$:\\\\
	$\frac{6}{3\cdot17}+\frac{7}{2\cdot17}+\frac{2}{4\cdot17}+\frac{2}{2\cdot17}=\frac{7}{17}$\\\\
	умножим обе части уравнения на 17:\\\\
	$
	\frac{6}{3}+\frac{7}{2}+\frac{2}{4}+\frac{2}{2}=7;\\\\
	2+3.5+0.5+1=7
	$\\\\
	ответ: $(\frac{6}{17},\frac{7}{17},\frac{2}{17},\frac{2}{17})$.
	\subsection*{2}
	обозначим стационарное распределение $\pi=(\pi_1,\pi_2,\pi_3,\pi_4)$;\\\\
	$
	\pi P=\pi;\\\\
	P^T\pi^T=\pi^T;\\\\
	P^T\pi^T-\pi^T=0;\\\\
	(P^T-E)\pi^T=0;
	$\\\\
	решим полученную систему уравнений методом Гаусса:\\
	запишем расширенную матрицу системы
	\[
	\begin{array}({@{}ccccc|c@{}})
		-1 & 0 & 1/5 & 0 & 0 & 0\\
		1/2 & -1 & 1/5 & 1/2 & 1/2 & 0\\
		0 & 1 & -4/5 & 0 & 1/2 & 0\\
		0 & 0 & 1/5 & -1 & 0 & 0\\
		1/2 & 0 & 1/5 & 1/2 & -1 & 0
	\end{array}\
	\]
	приведем ее к треугольному виду
	\[
	\begin{array}({@{}ccccc|c@{}})
		-1 & 0 & 1/5 & 0 & 0 & 0\\
		0 & -1 & 3/10 & 1/2 & 1/2 & 0\\
		0 & 1 & -4/5 & 0 & 1/2 & 0\\
		0 & 0 & 1/5 & -1 & 0 & 0\\
		0 & 0 & 3/10 & 1/2 & -1 & 0
	\end{array}\
	\]
	\[
	\begin{array}({@{}ccccc|c@{}})
		-1 & 0 & 1/5 & 0 & 0 & 0\\
		0 & -1 & 3/10 & 1/2 & 1/2 & 0\\
		0 & 0 & -1/2 & 1/2 & 1 & 0\\
		0 & 0 & 1/5 & -1 & 0 & 0\\
		0 & 0 & 3/10 & 1/2 & -1 & 0
	\end{array}\
	\]
	\[
	\begin{array}({@{}ccccc|c@{}})
		-1 & 0 & 1/5 & 0 & 0 & 0\\
		0 & -1 & 3/10 & 1/2 & 1/2 & 0\\
		0 & 0 & -1/2 & 1/2 & 1 & 0\\
		0 & 0 & 0 & -4/5 & 2/5 & 0\\
		0 & 0 & 0 & 4/5 & -2/5 & 0
	\end{array}\
	\]
	\[
	\begin{array}({@{}ccccc|c@{}})
		-1 & 0 & 1/5 & 0 & 0 & 0\\
		0 & -1 & 3/10 & 1/2 & 1/2 & 0\\
		0 & 0 & -1/2 & 1/2 & 1 & 0\\
		0 & 0 & 0 & -4/5 & 2/5 & 0\\
		0 & 0 & 0 & 0 & 0 & 0
	\end{array}\
	\]
	$
	\pi_4=\frac{\pi_5}{2}\\\\
	\pi_3=2(\frac{\pi_4}{2}+\pi_5)=\frac{5\pi_5}{2}\\\\
	\pi_2=\frac{3\pi_3}{10}+\frac{\pi_4}{2}+\frac{\pi_5}{2}=\frac{3\pi_5}{2}
	\pi_1=\frac{\pi_3}{5}=\frac{\pi_5}{2}
	$\\\\
	так как $\pi$ - распределение, то $\sum_{i=1}^{5}\pi_i=1$, тогда\\\\ $
	\pi_5(\frac{1}{2}+\frac{3}{2}+\frac{5}{2}+\frac{1}{2}+1)=1\\\\
	\pi_5=\frac{1}{6}
	$\\\\
	ответ: $\pi=(\frac{1}{12},\frac{1}{4},\frac{5}{12},\frac{1}{12},\frac{1}{6})$
	\section*{задание 1.12}
	\subsection*{1}
	обозначим состояния цепи как 0 и 1;\\
	так как $p_{00}=p_{11}=0$, то возможны переходы только из 0 в 1 и из 1 в 0, но тогда цепь периодична с периодом 2, поэтому неэргодична.
	\subsection*{2}
	обозначим состояния цепи как 0 и 1;\\
	так как $p_{01}=p_{10}=0$, то возможны переходы только из 0 в 0 и из 1 в 1, но тогда цепь разложима (так как состояние 1 недостижимо из состояния 0), поэтому неэргодична.
	\subsection*{3}
	обозначим состояния цепи как 0 и 1;\\
	так как $p_{01}=p_{11}=0$, то возможны переходы только из 0 в 0 и из 1 в 0, но тогда цепь разложима (так как состояние 1 недостижимо из состояния 0), поэтому неэргодична.
	\subsection*{4}
	обозначим состояния цепи как 0 и 1;\\
	так как $p_{11}=0,p_{01}\ne0,p_{10}\ne0$, то невозможен переход из 1 в 1, но возможен цикл $1\rightarrow0\rightarrow1$, значит состояние 1 периодично с периодом 2, поэтому цепь неэргодична.
	\subsection*{5}
	обозначим состояния цепи как 0 и 1;\\
	так как $p_{10}=0$, то невозможен переход из 1 в 0, то есть цепь разложима, поэтому неэргодична.
	\subsection*{6}
	обозначим состояния цепи как 0, 1, 2;\\\\
	так как $p_{00}\ne0,p_{11}\ne0,p_{22}\ne0$, то возможны переходы $0\rightarrow0$, $1\rightarrow1$, $2\rightarrow2$, значит цепь апериодическая (для любого состояния $i$ НОД периодов этого состояния равен 1, так как возможен переход $i\rightarrow i$);\\\\
	кроме того, так как $p_{01}\ne0,p_{12}\ne0,p_{21}\ne0$, то возможны переходы $0\rightarrow1$, $1\rightarrow2$, $2\rightarrow0$;\\
	получили, что любое состояние достижимо из любого, значит цепь неразложима;\\\\
	в этой цепи из состояния $i$ можно перейти в состояние $i$ за 1 (переход $i\rightarrow i$), 3 ($i\rightarrow (i+1)(mod 3)\rightarrow (i+2)(mod 3)\rightarrow i$) или $n>3$ шагов;\\
	переход из $i$ в $i$ за $n=3+a+b$ шагов означает, что в состоянии $j_1=(i+1)(mod 3)$ было сделано $a$ переходов $j_1\rightarrow j_1$ и в состоянии $j_2=(i+2)(mod 3)$ было сделано $b$ переходов $j_2\rightarrow j_2$;\\
	$b$ зависит от $a$: $b=n-3-a$;\\
	$a$ может принимать $n-2$ значений от 0 до $n-3$, то есть перейти из $i$ в $i$ за $n>3$ шагов можно $n-2$ способами;\\
	так как вероятность любого перехода в данной цепи равна $\frac{1}{2}$, то вероятность перехода из $i$ в $i$ за $n$ шагов $f_{ii}^{(n)}=\frac{1}{2^n}$;\\
	обозначим $T_i$ количество шагов для первого возвращения из $i$ в $i$ и найдем математическое ожидание $T_i$:\\\\
	$E[T_i]=\sum_{1}^{\infty}T_ip(T_i)=1\cdot\frac{1}{2}+3\cdot\frac{1}{8}+\sum_{4}^{\infty}\frac{n(n-2)}{2^n}$;\\\\
	применим признак Д'аламбера:\\\\ $\lim_{n\rightarrow\infty}\frac{(n+1)(n-1)}{2^{n+1}}\cdot\frac{2^n}{n(n-2)}=\frac{1}{2}<1$,\\\\
	поэтому ряд $\sum_{4}^{\infty}\frac{n(n-2)}{2^n}$ сходится, значит $E[T_i]<\infty$ и цепь положительно возвратна;\\\\
	ответ: цепь эргодична, так как она апериодична, неразложима и положительно возвратна.
	\end{large}
\end{document}