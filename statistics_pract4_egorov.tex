\documentclass[a4paper]{article}
\usepackage[utf8]{inputenc}
\usepackage[russian]{babel}
\usepackage[T1]{fontenc}
\usepackage{amsmath}
\usepackage{amsfonts}
\usepackage{amssymb}
\usepackage{graphicx}
\author{Борис Егоров}
\title{статистика практика 4, дополнительные задачи}
\begin{document}
	\maketitle
	\begin{large}
	\section*{2}
	обозначим $z=x+y$, тогда $F_z(t)=P(x+y<t)$;\\\\
	так как x дискретна, то по формуле полной вероятности получим:\\\\
	$
	P(x+y<t)=\sum_{i}P(x+y<t|x=a_i)P(x=a_i)=\\\\
	\sum_{i}P(a_i+y<t)p_i=\sum_{i}p_iP(y<t-a_i)=\sum_{i}p_iF_y(t-a_i)
	$;\\\\
	тогда\\\\
	$
	g(t)=\frac{dF_z(t)}{dt}=\frac{d\sum_{i}p_iF_y(t-a_i)}{dt}=\sum_{i}p_i\frac{dF_y(t-a_i)}{dt}=\sum_{i}p_if(t-a_i)
	$
	\end{large}
\end{document}