\documentclass[a4paper]{article}
\usepackage[utf8]{inputenc}
\usepackage[russian]{babel}
\usepackage[T1]{fontenc}
\usepackage{amsmath}
\usepackage{amsfonts}
\usepackage{amssymb}
\usepackage{graphicx}
\author{Борис Егоров}
\title{статистика практика 5, дополнительные задачи}
\begin{document}
	\maketitle
	\begin{large}
	\section*{1}
	\subsection*{математическое ожидание}
	$
	E(x)=\\\\
	p+2pq+3pq^2+...=\\\\
	p(q^0+2q^1+3q^2+...)=\\\\
	(1-q)(q^0+2q^1+3q^2+...)=\\\\
	(q^0+2q^1+3q^2+...)-q(q^0+2q^1+3q^2+...)=\\\\
	(q^0+2q^1+3q^2+4q^3...)-(q^1+2q^2+3q^3+...)
	$\\\\
	сгруппируем слагаемые в последнем выражении:\\\\
	$
	(q^0+2q^1+3q^2+4q^3...)-(q^1+2q^2+3q^3+...)=\\\\
	q^0+(2q^1-q^1)+(3q^2-2q^2)+(4q^3-3q^3)+...=\\\\
	q^0+q^1+q^2+q^3+...
	$\\\\
	получили сумму бесконечной убывающей геометрической прогрессии:\\\\
	$q^0+q^1+q^2+q^3+...=\frac{q^0}{1-q}=\frac{1}{1-q}=\frac{1}{p}$
	\subsection*{дисперсия}
	$
	D(x)=E(x^2)-(E(x))^2;\\\\
	E(x^2)=p+2^2pq+3^2pq^2+4^2pq^3+...=\\\\
	p(1+2^2q+3^2q^2+4^2q^3+...)=\\\\
	(1-q)(1+2^2q+3^2q^2+4^2q^3+...)=\\\\
	(1+2^2q+3^2q^2+4^2q^3+...)-q(1+2^2q+3^2q^2+4^2q^3+...)=\\\\
	(1+2^2q+3^2q^2+4^2q^3+...)-(q+2^2q^2+3^2q^3+...)=\\\\
	1+q(2^2-1)+q^2(3^2-2^2)+q^3(4^2-3^2)+...=\\\\
	1+q(2-1)(2+1)+q^2(3-2)(3+2)+q^3(4-3)(4+3)+...=\\\\
	1+q(2+1)+q^2(3+2)+q^3(4+3)+...=\\\\
	\sum_{k=0}^{+\infty}(2k+1)q^k=\\\\
	2\sum_{k=1}^{+\infty}kq^k+\sum_{k=0}^{+\infty}q^k=2S_1+S_2;\\\\
	S_1=\frac{1}{1-q};
	$\\\\
	так как $E=(1-q)\sum_{k=1}^{+\infty}kq^{k-1}$, то $S_1=\frac{qE}{1-q}=\frac{q}{(1-q)^2}$;\\\\
	получим $E(x^2)=2S_1+S_2=\frac{2q}{(1-q)^2}+\frac{1}{1-q}=\frac{1+q}{(1-q)^2}$;\\\\
	тогда $D(x)=E(x^2)-(E(x))^2=\frac{1+q}{(1-q)^2}-\frac{1}{(1-q)^2}=\frac{q}{(1-q)^2}=\frac{q}{p^2}$
	\section*{2}
	пусть $f(x)=0$ при $x<0$,\\\\
	$f(x)=\lambda e^{-\lambda x}$ при $x\notin\{(n-\frac{1}{n^2},n+\frac{1}{n^2})\},n\in\mathbb{N},n>n_0$,\\\\
	на каждом интервале $(n-\frac{1}{n^2},n)$ монотонно возрастает\\
	от $f(n-\frac{1}{n^2})$ до 1,\\\\
	на каждом интервале $(n,n+\frac{1}{n^2})$ монотонно убывает\\
	от 1 до $f(n+\frac{1}{n^2})$;\\\\
	можно так подобрать $\lambda$ и $n_0$, что интеграл $\int_{-\infty}^{+\infty}f(x)dx$ будет равен 1\\\\
	ответ: неправда
	\end{large}
\end{document}